\documentclass[titlepage, 12pt, a4paper, oneside]{article}
\usepackage[utf8]{inputenc}
\usepackage[spanish, es-tabla]{babel}
\usepackage[T1]{fontenc}
\usepackage{hyperref}
\usepackage[numbib]{tocbibind}
\usepackage{tikz}
\usepackage[top=1in, bottom=1.25in, left=1.25in, right=1.25in]{geometry}
\usepackage{xcolor}

\usepackage{fancyhdr}
\pagestyle{fancy}
\fancyhf{}
\rhead{\textit{\color[rgb]{0.0,0.424,0.616}Nombre del estudiante}}
\lhead{}
\rfoot{}
\renewcommand{\headrulewidth}{0pt}

\usepackage{pgfgantt}

\title{}
\date{}
\renewcommand{\familydefault}{\sfdefault}

\begin{document}
\thispagestyle{empty}
\tikz[remember picture,overlay] \node[opacity=1.0,inner sep=0pt] at (current page.center){\includegraphics[width=\paperwidth,height=\paperheight]{Plantilla_AnteProyectoTFG-portada}};

\begin{center}
  \vspace{4cm}
  {\color{white} \Huge \textbf{Mezclador en Red para el Intercomunicador intercomTM}}
\end{center}

\Large

\vspace{16.5ex}
\begin{tabular}{ll}
  ~~~~~~~~~ & Nombre del estudiante
\end{tabular}

\vspace{1.2cm}
\begin{tabular}{ll}
  ~~~~~~~~~ & Grado en Ingeniería Informática
\end{tabular}

\vspace{1.1cm}
\begin{tabular}{ll}
  ~~~~~~~~~ & Vicente González Ruiz
\end{tabular}

\vspace{1.2cm}
\begin{tabular}{ll}
  ~~~~~~~~~ & Departamento de Informática
\end{tabular}

\vspace{0.95cm}
\begin{tabular}{ll}
  ~~~~~~~~~ & Trabajo Técnico
\end{tabular}

\vspace{0.95cm}
\begin{tabular}{ll}
  ~~~~~~~~~ & Paralelismo de Tareas, Audio, Intercomunicador, \\
  ~~~~~~~~~ & Procesamiento en Tiempo Real
\end{tabular}

\clearpage

\tikz[remember picture,overlay] \node[opacity=1.0,inner sep=0pt] at (current page.center){\includegraphics[width=\paperwidth,height=\paperheight]{Plantilla_AnteProyectoTFG-paginas}};

\normalsize

\section{Introducción}
\label{sec:intro}
Un intercomunicador (intercom en inglés), es un dispositivo que
permite a dos o más usuarios comunicarse en tiempo real. Normalmente
sólo transmiten sonido, aunque también pueden transmitir imágenes
(vídeos). Se usan en los portales de las casas para saber quién está
llamando en la puerta, lo usan los motoristas para hablar entre ellos
cuando se desplazan y (lógicamente) llevan cascos, y se usan en
aplicaciones de comunicación a través de Internet, como Skype,
WhatsApp, etc.

El intercomunicador intercomTM es una aplicación desarrollada en la
Universidad de Almería por los alumnos de la asignatura de Tecnologías
Multimedia~\cite{intercomTM}, y posee las siguientes características:
\begin{enumerate}
\item Permite transmitir audio mono y estereo.
\item Permite elegir la frecuencia de muestreo.
\item Se adapta al ancho de banda disponible, enviando una señal de
  audio de calidad proporcional al ancho de banda.
\item Permite controlar la latencia.
\item Puede usarse en Internet.
\end{enumerate}

\section{Objetivos y justificación}
\label{sec:objetivos}
El objetivo de este TFG es diseñar e implementar un mezclador de audio
para el intercomunicador intercomTM. Con la funcionalidad actual es
posible que dos o más usuarios de intercomTM puedan comunicarse entre
sí, lanzando varias instancias de intercomTM, una para cada posible
pareja de usuarios. Sin embargo, esta solución presentaría dos
inconvenientes:
\begin{enumerate}
\item El sistema operativo debería permitir que al menos dos instancias
  de intercomTM usen el hardware de audio simultaneamente. Este es un
  requerimiento que no siempre puede cumplirse.
\item Los usuarios enviarían tantas copias del stream de audio como
  interlocutores hay en la red. Esto significa que la redundancia de
  datos en el enlace de salida de los usuarios de intercomTM podría
  ser alta.
\end{enumerate}

\begin{figure}
  \begin{center}
    \includegraphics[width=5cm]{comparativa}
  \end{center}
  \caption{Flujo de datos entre intercomunicadores ($I$), cuando no se
    usa (izquerda) y sí se usa (derecha) un mezclador ($M$).}
  \label{fig:comparativa}
\end{figure}

Ambos problemas pueden solucionarse si los interlocutores no se
comunican directamente, sino que envían una única copia de su stream a
un servidor instalado en la red y reciben de él una única copia del
stream resultante de la mezcla de todos los stream que llegan a dicho
servidor. Dicho servidor funciona, básicament, como un mezclador de
audio y, lo denominaremos, por tanto, mixerTM. En la
Figura~\ref{fig:comparativa} se muestra un ejemplo que muestra, para 3
interlocutores, cómo serían los flujos de datos entre los diferentes
intercomunicadores con y sin y el mezclador. Como puede verse, cuando
no existe el mezclador (izquierda), cada intercomunicador (que en
realidad serían dos instancias de intercomTM, ejecutándose en
paralelo) debe enviar dos copias del stream, y también recibe dos
copias. Con el mezclador (derecha), cada intercomunicador (que ahora
sí es una única instancia de intercomTM) envía una única copia y
recibe una sola copia. Es el mezcador quien se encarga de recibir
todos los streams, mezclarlos, y reenviarlos a cada uno de los
intercomunicadores.

\section{Fases de desarrollo y estimación de su coste temporal}
\label{sec:fases}
Se preveen las siguientes unidades de trabajo y temporizaciones (véase
la Figura~\ref{fig:temporizacion}):
\begin{enumerate}
  \item {IDENTIFICACIÓN}: Identificación y comprensión de los
    requerimientos y objetivos planteados (10 h).
  \item {ANÁLISIS:} Estudio y análisis de la implementación actual de
    intercomTM, comprendiendo los algoritmos que utiliza (60 h).
  \item {APROVISIONAMIENTO}: Instalación y configuración del entorno
    de desarrollo y ejecución. Básicamente, Linux y Python (10h).
  \item {IMPLEMENTACIÓN}: Implementación del mezclador (60h).
  \item {EVALUACIÓN}: Evaluación de la solución y cuantificación del
    grado de consecución de los objetivos (30h).
  \item {REDACCIÓN}: Redacción de la memoria (30h).
\end{enumerate}

\begin{figure}
  \begin{center}
    \resizebox{\columnwidth}{!}{
      \begin{ganttchart}{1}{25}{10}
        \gantttitle{Bloques de 10 horas (total, 250 h)}{25} \\
        \gantttitlelist{1,...,25}{1} \\
        \ganttbar{IDENTIFICACIÓN}{1}{2} \\ % 50h
        \ganttlinkedbar{ANÁLISIS}{3}{9} \\ % 50 h
        \ganttlinkedbar{APROVISIONAMIENTO}{10}{11} \\ % 50 h
        \ganttlinkedbar{IMPLEMENTACIÓN}{12}{18} \\ % 50 h
        \ganttlinkedbar{EVALUACIÓN}{19}{22} \\ % 50 h
        \ganttlinkedbar{REDACCIÓN}{23}{25} % 50 h
      \end{ganttchart}
    }
  \end{center}
  \caption{Temporización del TFG.\label{fig:temporizacion}}
\end{figure}

\section{Requerimientos}
\label{sec:requerimientos}
Los requerimientos del mezclador son:

\begin{itemize}
\item Generales:
  \begin{enumerate}
  \item El lenguaje de programación será Python, y en concreto, se
    hará uso de la bilioteca multiprocessing~\cite{multiprocessing}.
  \item El estilo de programación debe seguir la norma
    PEP-8~\cite{PEP8}.
  \item El idioma usado, tanto para la definición de objetos, métodos,
    variables, etc., como para la documentación (incluyendo los
    comentarios en el código fuente) debe ser inglés.
  \item Se usará el paragigma de programación orientado a
    objetos~\cite{schach2008object} para ir obteniendo diferentes
    aproximaciones sucesivas del resultado final.
  \item El producto debe ejecutarse con éxito, al menos, en Linux,
    Windows y OSX.
  \end{enumerate}
\item Específicos:
  \begin{enumerate}
  \item El mezclador usará paralelismo de
    tareas~\cite{pacheco2011introduction} (un proceso por cada
    intercomunicador conectado, más otro proceso que realizará la
    mezcla y realizará el envío).
  \item El mezclador realizará control de flujo de datos con los
    intercomunicadores conectados.
  \end{enumerate}
\end{itemize}

\section{Resultados esperados}
\label{sec:resultados}
De este trabajo debería resultar una implementación del mezclador
mixerTM, que debería satisfacer los requerimientos plantados. Se
llevarán a cabo una serie de experimentos para verificar su
funcionamiento correcto bajo diferentes cargas de trabajo, variando
las tasas de muestreo, el número de canales, el ancho de banda
disponible, el número de intercomunicadores conectados, etc.

\section{Conclusiones esperadas}
\label{sec:conclusiones}
Tras un análisis de los resultados obtenidos, debería determinarse las
características básicas (cantidad de memoria, número de cores, ancho
de banda disponible, etc.) del host en el que corre el
mezclador. También debería concluirse si el uso del mezclador influye
significativamente en la latencia.

Básicamente, se tratará de concluir si el modelo de paralelismo de
datos proporcionado por Python a través de su API
multiprocessing~\cite{multiprocessing} es adacuado (o no) para
conseguir los objetivos definidos (véase la
Sección~\ref{sec:objetivos}).

\bibliographystyle{plain}
\bibliography{../bibliografia}

\begin{center}
  \textbf{Firma del director (codirector)}
\end{center}

\end{document}
